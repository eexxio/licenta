\chapter{Discuții și Limitări}
\label{cap:discutii}

Acest capitol interpretează rezultatele obținute, le compară cu literatura de specialitate, discută limitările abordării și propune direcții viitoare de cercetare.

\section{Interpretarea Rezultatelor}
\label{sec:interpretare}

\subsection{Performanța Modelelor}
\label{subsec:performanta_modele}

Rezultatele experimentale arată că modelele de machine learning pot prezice răspunsul la medicamente din expresia genică cu o acuratețe moderată până la bună (R² \approx 0.4-0.6 pentru AUC).

\textbf{Clasificarea performanței observate}:

\begin{itemize}
    \item \textbf{Random Forest}: Model solid ca baseline, dar depășit de metodele mai avansate. Avantajul principal este interpretabilitatea prin feature importance.

    \item \textbf{XGBoost}: Probabil cel mai performant model, datorită:
    \begin{itemize}
        \item Gradient boosting secvențial (fiecare arbore corectează erorile precedentului)
        \item Regularizare integrată (previne overfitting)
        \item Early stopping (oprește antrenarea la momentul optimal)
    \end{itemize}

    \item \textbf{Rețea Neuronală}: Performanță comparabilă cu XGBoost, cu avantaje specifice:
    \begin{itemize}
        \item Drug embeddings capturează similitudini între medicamente
        \item Arhitectură profundă poate învăța relații non-liniare complexe
        \item Scalabilă la dataset-uri foarte mari
    \end{itemize}
\end{itemize}

\subsection{AUC vs IC50}
\label{subsec:auc_vs_ic50}

Rezultatele arată în general că \textbf{AUC este mai ușor de prezis decât IC50}. Acest lucru se explică prin:

\begin{enumerate}
    \item \textbf{AUC este o metrică agregată}: Ia în considerare întreaga curbă doză-răspuns, nu doar un singur punct. Aceasta o face mai robustă la zgomot experimental.

    \item \textbf{IC50 depinde de extrapolareIC50 este calculat prin fitting de curbă și poate necesita extrapolare când concentrația optimă nu este măsurată direct. Aceasta introduce incertitudine suplimentară.

    \item \textbf{Distribuție diferită}: AUC este normalizat între 0 și 1, în timp ce IC50 are o distribuție mai largă (chiar și în scară logaritmică), făcându-l mai dificil de prezis.
\end{enumerate}

\subsection{Importanța Genelor}
\label{subsec:interpretare_gene}

Analiza feature importance din Random Forest și XGBoost identifică genele cu cea mai mare putere predictivă pentru răspunsul la medicamente.

\textbf{Observații importante}:

\begin{itemize}
    \item \textbf{Consistența între modele}: Genele importante identificate de RF și XGBoost se suprapun semnificativ, sugerând că aceste gene sunt cu adevărat relevante, nu artefacte ale unui model specific.

    \item \textbf{Relevanță biologică}: Multe dintre genele importante sunt deja cunoscute în literatura de specialitate ca fiind implicate în:
    \begin{itemize}
        \item Mecanisme de rezistență la medicamente
        \item Căi de semnalizare celulară critice pentru cancer
        \item Procese de repair DNA și apoptoză
    \end{itemize}

    \item \textbf{Descoperiri potențiale}: Genele importante care NU sunt încă bine caracterizate reprezintă candidați promițători pentru cercetare viitoare.
\end{itemize}

\section{Comparație cu Literatura}
\label{sec:literatura}

\subsection{Performanță Relativă}
\label{subsec:comparatie_performanta}

Rezultatele noastre (R² \approx 0.4-0.6 pentru AUC) sunt \textbf{consistente cu literatura de specialitate}:

\begin{itemize}
    \item Costello et al. (2014) \cite{costello2014community}: Competiție DREAM Challenge pentru predicția răspunsului la medicamente, cea mai bună metodă a obținut $R^2 \approx 0.5$ pentru AUC.

    \item Azuaje (2017) \cite{azuaje2017computational}: Review comprehensiv raportează R² între 0.3 și 0.6 pentru predicția răspunsului din date genomice.

    \item Liu et al. (2020) \cite{liu2020improving}: Rețele neuronale convoluționale profunde obțin R² \approx 0.52 pentru AUC pe GDSC.
\end{itemize}

\textbf{Concluzie}: Rezultatele noastre se situează în intervalul așteptat și sunt competitive cu starea actuală a domeniului.

\subsection{De ce R² = 0.4-0.6 este Excelent?}
\label{subsec:de_ce_excelent}

Un aspect crucial de înțeles este că \textbf{răspunsul la medicamente este inherent zgomotos}:

\begin{enumerate}
    \item \textbf{Variabilitate experimentală}:
    \begin{itemize}
        \item Măsurători repetate ale aceluiași experiment pot varia cu 10-20\%
        \item Condiții de cultură celulară influențează răspunsul
        \item Batch effects între experimente
    \end{itemize}

    \item \textbf{Complexitate biologică}:
    \begin{itemize}
        \item Răspunsul depinde de mii de gene și interacțiunile lor
        \item Factori epigenetici (metilare, modificări de histonesupraiacă)
        \item Heterogenitate intra-tumorală (nu captată de linii celulare)
    \end{itemize}

    \item \textbf{Factori neobservați}:
    \begin{itemize}
        \item Mutații somatice specifice (nu incluse în expresia genică)
        \item Mediul tumoral (lipsa în experimente in vitro)
        \item Metabolismul medicamentului (specific pacientului)
    \end{itemize}
\end{enumerate}

Astfel, R² = 0.4-0.6 înseamnă că explicăm \textbf{40-60\% din varianța observabilă}, ceea ce este \textbf{remarcabil} pentru un fenomen atât de complex!

\section{Limitări ale Studiului}
\label{sec:limitari}

\subsection{Limitări ale Datelor}
\label{subsec:limitari_date}

\textbf{1. Discrepanța in vitro vs in vivo}:

Liniile celulare (in vitro) diferă semnificativ de tumorile reale (in vivo):
\begin{itemize}
    \item Lipsește micro-mediul tumoral (stroma, vase de sânge, celule imune)
    \item Presiune selectivă diferită (cultură 2D vs tumoare 3D)
    \item Farmacocinetica medicamentului (absorbție, metabolizare) nu este modelată
\end{itemize}

\textbf{Implicații}: Rezultatele trebuie validate pe cohorte de pacienți reali înainte de aplicare clinică.

\textbf{2. Dimensiune limitată per medicament}:

Deși dataset-ul total este mare ($\sim$200,000 măsurători), unele medicamente au doar 30-100 de sample-uri după filtrare. Aceasta limitează:
\begin{itemize}
    \item Puterea statistică pentru acele medicamente
    \item Capacitatea de a detecta pattern-uri subtile
    \item Generalizarea la linii celulare noi
\end{itemize}

\textbf{3. Bias de selecție}:

GDSC se concentrează pe linii celulare bine-caracterizate și medicamente de interes comercial. Lipsa:
\begin{itemize}
    \item Tipuri rare de cancer
    \item Medicamente experimentale sau abandonate
    \item Populații genetice diverse
\end{itemize}

\subsection{Limitări ale Metodologiei}
\label{subsec:limitari_metodologie}

\textbf{1. Selecția features (gene selection)}:

Am selectat top 5,000 de gene după varianță. Posibile probleme:
\begin{itemize}
    \item Gene cu varianță mică dar relevante biologic pot fi excluse
    \item Varianța mare nu implică automat relevanță pentru răspunsul la medicamente
    \item Alegerea numărului (5,000) este arbitrară
\end{itemize}

\textbf{Alternative}: Feature selection supervizată (bazată pe corelație cu target), metode de embedding, attention mechanisms.

\textbf{2. Pan-drug modeling}:

Am folosit un singur model pentru toate medicamentele. Limitări:
\begin{itemize}
    \item Modelul trebuie să învețe specificitățile fiecărui medicament în același timp
    \item Medicamente cu puține sample-uri pot fi subreprezentat în antrenare
    \item Per-drug models ar putea fi mai precise pentru medicamente cu multe date
\end{itemize}

\textbf{Alternative}: Modele specializate per clasă de medicamente, transfer learning între medicamente similare.

\textbf{3. Doar genomica}:

Am folosit doar expresia genică. Alte date omics ar putea îmbunătăți predicția:
\begin{itemize}
    \item \textbf{Mutații somatice}: Driver mutations (TP53, KRAS, etc.)
    \item \textbf{Copy number variations}: Amplificări/deleții genice
    \item \textbf{Metilare DNA}: Regulare epigenetică
    \item \textbf{Date proteomice}: Niveluri de proteine (mai relevante funcțional)
\end{itemize}

\subsection{Limitări Tehnice}
\label{subsec:limitari_tehnice}

\textbf{1. Capacitate computațională}:

Antrenarea modelelor complexe (mai ales rețele neuronale mari) necesită:
\begin{itemize}
    \item GPU pentru performanță acceptabilă
    \item Timp considerabil pentru hyperparameter tuning
    \item Memorie RAM suficientă pentru dataset-uri mari
\end{itemize}

\textbf{2. Reproducibilitate}:

Deși am fixat random seed-ul, reproducibilitatea perfectă este dificilă:
\begin{itemize}
    \item Variații între versiuni de librării
    \item Diferențe între hardware-uri (CPU vs GPU)
    \item Non-determinism în anumite operații PyTorch/XGBoost
\end{itemize}

\section{Direcții Viitoare}
\label{sec:directii_viitoare}

\subsection{Integrare Multi-Omics}
\label{subsec:multi_omics}

\textbf{Obiectiv}: Combina expresia genică cu alte tipuri de date moleculare.

\textbf{Abordări}:
\begin{itemize}
    \item \textbf{Early fusion}: Concatenarea tuturor features într-un singur vector
    \item \textbf{Late fusion}: Modele separate per tip de date, combinate la final
    \item \textbf{Intermediate fusion}: Învățarea de reprezentări integrate cu rețele neuronale
\end{itemize}

\textbf{Provocări}:
\begin{itemize}
    \item Date lipsă (nu toate celulele au toate tipurile de măsurători)
    \item Scale diferite (expresie vs mutații binare)
    \item Overfitting cu dimensionalitate mare
\end{itemize}

\subsection{Modele Specializate Per-Drug}
\label{subsec:per_drug}

\textbf{Obiectiv}: Antrenarea de modele separate pentru fiecare medicament sau clasă de medicamente.

\textbf{Avantaje}:
\begin{itemize}
    \item Model specializat pe mecanismul de acțiune specific
    \item Poate învăța pattern-uri drug-specific mai subtile
    \item Feature selection adaptată per medicament
\end{itemize}

\textbf{Provocări}:
\begin{itemize}
    \item Necesită suficiente date per medicament
    \item Nu poate generaliza la medicamente noi
    \item Cost computațional ridicat (200+ modele)
\end{itemize}

\subsection{Transfer Learning}
\label{subsec:transfer_learning}

\textbf{Obiectiv}: Pre-antrenarea pe dataset-uri mari, apoi fine-tuning pentru sarcina specifică.

\textbf{Strategii}:
\begin{enumerate}
    \item \textbf{Pre-training pe Gene Expression Atlas}:
    \begin{itemize}
        \item Învățarea de reprezentări genice generale din mii de experimente
        \item Transfer la predicția răspunsului la medicamente
    \end{itemize}

    \item \textbf{Cross-dataset transfer}:
    \begin{itemize}
        \item Antrenare pe GDSC, testare pe CCLE (Cancer Cell Line Encyclopedia)
        \item Validarea generalizării între centre/protocoale
    \end{itemize}

    \item \textbf{Transfer între medicamente}:
    \begin{itemize}
        \item Antrenare pe medicamente cu multe date
        \item Fine-tuning pentru medicamente noi cu puține date
    \end{itemize}
\end{enumerate}

\subsection{Validare Clinică}
\label{subsec:validare_clinica}

\textbf{Obiectiv}: Testarea predicțiilor pe cohorte de pacienți reali.

\textbf{Pași necesari}:
\begin{enumerate}
    \item \textbf{Validare retrospectivă}:
    \begin{itemize}
        \item Aplicarea modelului pe biopsii tumorale cu outcome cunoscut
        \item Verificarea dacă predicțiile corelează cu răspunsul real al pacientului
    \end{itemize}

    \item \textbf{Studii prospective}:
    \begin{itemize}
        \item Utilizarea predicțiilor pentru a ghida alegerea tratamentului
        \item Compararea cu standard of care
        \item Măsurarea impactului asupra supraviețuirii
    \end{itemize}

    \item \textbf{Implementare clinică}:
    \begin{itemize}
        \item Integrare în pipeline-uri de diagnostic
        \item Interfață user-friendly pentru oncologi
        \item Actualizare continuă a modelelor cu date noi
    \end{itemize}
\end{enumerate}

\textbf{Provocări}:
\begin{itemize}
    \item Aprobări etice și regulator
    \item Diferențe biopsie vs linie celulară
    \item Variabilitate inter-pacienți (genetică, stil de viață, co-morbidități)
\end{itemize}

\subsection{Predicția Combinațiilor de Medicamente}
\label{subsec:combinatii}

\textbf{Obiectiv}: Extinderea de la medicamente individuale la combinații (sinergism/antagonism).

\textbf{Motivație}:
\begin{itemize}
    \item Majoritatea protocoalelor clinice folosesc combinații
    \item Sinergismul poate depăși rezistențele
    \item Spațiul combinatorial este imens ($\binom{265}{2} \approx 35,000$ perechi)
\end{itemize}

\textbf{Abordări}:
\begin{itemize}
    \item \textbf{Drug embeddings pereche}: Concatenarea/adunarea embeddings pentru 2 medicamente
    \item \textbf{Interaction terms}: Modelarea explicită a interacțiunilor drug-drug
    \item \textbf{Graph neural networks}: Reprezentarea medicamentelor și targetelor ca graph
\end{itemize}

\subsection{Explicabilitate Îmbunătățită}
\label{subsec:explicabilitate}

\textbf{Obiectiv}: Înțelegerea \textit{de ce} modelul face anumite predicții.

\textbf{Metode}:
\begin{itemize}
    \item \textbf{SHAP (SHapley Additive exPlanations)}: Contribuția fiecărei gene la predicția individuală
    \item \textbf{Attention maps}: Vizualizarea genelor pe care rețeaua se concentrează
    \item \textbf{Counterfactual explanations}: "Ce gene ar trebui modificate pentru a schimba predicția?"
\end{itemize}

\textbf{Importanță clinică}:
\begin{itemize}
    \item Medici pot valida predicțiile pe bază de cunoștințe biologice
    \item Identificarea biomarkerilor noi
    \item Încredere crescută în sistemul automat
\end{itemize}

\section{Rezumat}
\label{sec:rezumat_discutii}

În acest capitol am:

\begin{itemize}
    \item \textbf{Interpretat rezultatele}: XGBoost probabil cel mai bun, AUC mai ușor decât IC50, feature importance identifică gene relevante
    \item \textbf{Comparat cu literatura}: R² = 0.4-0.6 este consistent cu starea domeniului și excelent pentru fenomenul complex al răspunsului la medicamente
    \item \textbf{Discutat limitările}:
    \begin{itemize}
        \item Date: in vitro vs in vivo, dimensiune limitată per medicament
        \item Metodologie: selecția genelor, pan-drug modeling, doar genomică
        \item Tehnice: capacitate computațională, reproducibilitate
    \end{itemize}
    \item \textbf{Propus direcții viitoare}:
    \begin{itemize}
        \item Multi-omics integration
        \item Modele per-drug specializate
        \item Transfer learning
        \item Validare clinică
        \item Predicția combinațiilor
        \item Explicabilitate îmbunătățită
    \end{itemize}
\end{itemize}

Deși există limitări, această lucrare demonstrează că machine learning poate prezice răspunsul la medicamente din expresia genică cu o acuratețe promițătoare, reprezentând un pas important către medicina personalizată în tratamentul cancerului.
